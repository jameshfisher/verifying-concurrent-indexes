\subsection{Concurrent index algorithms}

The \enquote{array} approach to indexing,
  with one \enquote{slot} for every possible key in the universe of keys,
  would provide this,
  but carries its usual disadvantages.
The hash table might provide a similar level of concurrency---one
  \IC{remove} operation per hash value---but again carries its usual disadvantages.

What about the \BST?
We might expect highly concurrent
  \IC{search},
  as this does not mutate the tree.
Our \IC{insert} operations may also be,
  as they only mutate leaves.
The \IC{remove} operation only mutates
  the removed node and its in-order predecessor,
  and so might only lock these.
It seems the \BST offers good support for highly concurrent operations.

Can this support carry through to balanced tree algorithms?
The basic reason the \BST is promising is that
  the tree is highly static and operations are highly \enquote{local}.
We can, then, dismiss variations like the splay tree,
  where every operation performs rotations
    at every level down to the key under consideration,
  which must lead to horrendous locking considerations
  (and to my knowledge, no concurrent splay tree algorithms have been attempted).
But at first glance,
  \emph{all} strictly balanced trees have the same problem:
  maintaining the invariant that promises balance
  can require fixing-up the tree all the way along one path from the root to a leaf.

The way most concurrent balanced trees tackle this
  is to \enquote{relax} the balancing invariant:
  simply, the tree is allowed to be in states that break the guarantee of balance.
The balancing logic is moved into a separate operation
  which can be performed independently of the \API operations---\eg, in a separate thread.

%% \begin{description}

%% \item[AVL tree]
%% Many concurrent AVL tree algorithms exist.
%% \cite{bronson_avl} proposes a relaxed


%% \item[Red-Black tree]
%% \ldots

%% \end{description}


\subsection{Concurrent Red-Black trees}

There’s more than one way to make a ‘concurrent RB tree’.


\subsubsection{Chromatic trees}

Introduced by Nurmi and Soisalon-Soininen in 1991/95.
Their tree only stores values in the leaves,
  and the internal nodes are merely ‘routers’.\footnotemark
Their paper uses the red/black edge interpretation of the RB tree.

\footnotetext{This seems like a really important change!  Is there an isomorphism between a standard RB-tree and one that stores everything in the leaves?  And how important is this from a performance POV?}


Instead of two states red/black,
  each edge is given a weight (an unsigned integer).
Red=0 and black=1.  Values < 0 impossible.  Values > 1 termed ‘overweighted’.

The \enquote{equal number of black nodes on any path} rule
  is transformed into:
  \enquote{equal sum of weights on any path}.
Without overweighting,
  this is equivalent (due to the values of red/black).

The \enquote{no two red nodes} rule is transformed into:
\enquote{the parent edges of the leaves are not red}.

The data structure spec now provides no guarantees of balancedness.
The two rules of the sequential RB tree together guarantee that
  longest path length / shortest path length <= 2.
The chromatic tree may have any lengths,
  in the case that all weights are zero (i.e. all edges are \enquote{red}) except for those of the leaves (which >0).

Note also that the B-tree isomorphism also disappears.
(Or does it?  In an arbitrarily-long run of reds, is there an analogous B-tree?)

Insertion/deletion perform no rebalancing,
  and maintain the CRB properties.
However, they may introduce violations of the red rule (consecutive reds).

Rebalancing is done by a separate thread.
Where the sequential RB tree balancing is done bottom-up
  (we insert the new node at a leaf,
  then rebalance moving towards the root),
  the chromatic rebalancing algorithm is top-down.

Concurrency control is done by three kinds of lock: r, w, x.
For any given node, the following patterns are possible:

\begin{tabular}{lll}
R       & W & X \\
0       & 0 & 0 \\
1...inf & 0 & 0 \\

\end{tabular}

Each thread, whether reading, writing or updating,
  locks nodes as it traverses the tree.
Rebalancing is done purely by changing node contents.
This is the locking scheme of Ellis for AVL trees\cite{ellis},
  with modifications.
E.g., Ellis proposes that writers w-lock the entire path,
  so that global balancing can be done;
  the decoupling makes w-lock coupling sufficient.


\subsubsection{Larsen's tree\cite{larsen}}

This builds on the chromatic tree,
  with simplifications to the rebalancing.
It requires a \enquote{problem queue}.
Hanke finds that Larsen’s tree,
  due to the queue and due to bottom-up conflict handling,
  does not perform as well as the chromatic tree.


\subsubsection{Hyperred-Black trees}

Gabarro et al criticize the Chromatic tree:
  in a run of red nodes, only the top pair may be updated.
This is because the rebalancing rules require that
  the grandparent is black (recall that this is a guarantee with the sequential RB tree).
Runs of red nodes may well happen:
  sorted insertion is common,
  and all new nodes are colored red.
(BUT in the chromatic tree paper above, leaf nodes (parent edges) are black!)
The suggested change is that
  as well as being hyperblack (‘overweighted’),
  nodes may be hyperred (‘underweighted’),
  i.e., weights <= 0 are degrees of red,
  and weights >= 1 are degrees of black.
The ‘blackness’ definition is maintained.
The authors have proofs of correctness.
However, they have experimental results
  that show a less-than-impressive performance vs. the chromatic tree.
