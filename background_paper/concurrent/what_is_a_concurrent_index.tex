\subsection{What is a concurrent index?}

The previous algorithms and data structures
  are designed for sequential access.
Performing concurrent operations on them results in undefined behavior.

One naive implementation of a concurrent index
  might simply enforce that, at any one time,
  at most one operation from $\Set{\IC{search}, \IC{insert}, \IC{remove}}$ is running.
A slightly better implementation
  might restrict to either any number of \IC{read}s,
  or one of $\Set{\IC{insert}, \IC{remove}}$.

This is still too restrictive:
  looking at the various indexing algorithms,
    many of them should, with minimal modification, permit
    multiple concurrent \IC{insert} operations---the \BST, for example.
With no knowledge of implementation,
  what restrictions should we expect from an index?
We might make the observation that, conceptually,
  the keys in an index are independent.
The fact that key \IC{k} is in the index or not,
  or that it has some value in the index,
  provides no information about key \IC{l}
  (where $\IC{k} ≠ \IC{l}$).
Therefore, we might expect restrictions to be placed only individual keys,
  rather than on the entire index.
For example, we might have the only restriction that,
  for any key \IC{k},
  at most one operation from $\Set{\IC{search(k)}, \IC{insert(k, \_)}, \IC{remove(k)}}$ is running.


\subsubsection{A concurrent index API specification}

\VEm
\noindent
\begin{tabular*}{\textwidth}{l l l}

% search where key is in the index
\Cond{\Pred{in}{\IC{i}, \IC{k}, v}}  &
\IC{r = i->search(k);}  &
\Cond{\Pred{in}{\IC{i}, \IC{k}, v} \Fand \IC{r} = v}
\\[0.5em]

% search where key is not in the index
\Cond{\Pred{out}{\IC{i}, \IC{k}}}  &
\IC{r = i->search(k);}  &
\Cond{\Pred{out}{\IC{i}, \IC{k}} \Fand \IC{r} = \nullref}
\\[0.5em]

% insert where key is in the index
\Cond{\Pred{in}{\IC{i}, \IC{k}, v\prime}}  &
\IC{i->insert(k, v);}  &
\Cond{\Pred{in}{\IC{i}, \IC{k}, v\prime}}
\\[0.5em]

% insert where key is not in the index
\Cond{\Pred{out}{\IC{i}, \IC{k}}}  &
\IC{i->insert(k, v);}  &
\Cond{\Pred{in}{\IC{i}, \IC{k}, \IC{v}}}
\\[0.5em]

% remove where key is in the index
\Cond{\Pred{in}{\IC{i}, \IC{k}, v}}  &
\IC{i->remove(k);}  &
\Cond{\Pred{out}{\IC{i}, \IC{k}}}
\\[0.5em]

% remove where key is not in the index
\Cond{\Pred{out}{\IC{i}, \IC{k}}}  &
\IC{i->remove(k);}  &
\Cond{\Pred{out}{\IC{i}, \IC{k}}}

\end{tabular*}
\VEm
