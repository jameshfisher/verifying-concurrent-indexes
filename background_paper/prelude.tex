\usepackage{xspace,listings,csquotes,url,cite}

\usepackage[a4paper]{geometry}


\newcommand{\VEm}{\vspace{1em}}


%%%%%%%%%%%%%%%%%%%%%
%%% Abbreviations %%%

\directlua{dofile("defs.lua") }


%%%%%%%%%%%%%%%%%%%%%%%%%%%%%%%%%%%%
%%% Project-specific math macros %%%

\newcommand{\Exists}[1]{\exists#1.\ }

\newcommand{\NewTerm}[1]{\textit{#1}\xspace} % put the term in the Index?

\newcommand{\Cond}[1]{\textcolor{blue}{\begin{math}\left\{#1\right\}\end{math}}\xspace}
\newcommand{\TermCond}[1]{\textcolor{blue}{\begin{math}\left[#1\right]\end{math}}\xspace}

\newcommand{\Triple}[3]{\Cond{#1}\ \IC{#2}\ \Cond{#3}\xspace}
\newcommand{\TermTriple}[3]{\TermCond{#1}\ \IC{#2}\ \TermCond{#3}\xspace}

\newcommand{\Pred}[2]{\mathsf{#1}(#2)\xspace}
\newcommand{\KeysOf}[1]{\mathsf{Keys}(#1)\xspace}

% Abbrevs
\newcommand{\CAP}{Concurrent Abstract Predicates\xspace}

\newcommand{\BST}{\textsc{BST}\xspace}
\newcommand{\BPlusT}{B\textsuperscript{+}-tree\xspace}
\newcommand{\BStarT}{B*-tree\xspace}
\newcommand{\RBt}{Red-Black tree\xspace}
\newcommand{\RBts}{Red-Black trees\xspace}

\newcommand{\ADT}{\textsc{ADT}\xspace}
\newcommand{\API}{\textsc{API}\xspace}

\newcommand{\strong}[1]{\textbf{#1}}
\newcommand{\lacuna}{\hspace{0.5em}\textrm{[\ldots]}\hspace{0.5em}}

% `Inline Code'
\newcommand{\IC}[1]{\texttt{#1}}

\usepackage[fleqn]{amsmath}

%% Types
% =====

% Type macro
\newcommand{\type}[1]{\ensuremath{\mathsf{#1}}}
% Subscripted type
\newcommand{\sst}[2]{{\ensuremath{{#1}_{#2}}}}

% Alphabet
\newcommand{\Talpha}{\ensuremath{\Sigma}}
% Hole Alphabet
\newcommand{\Tholes}{\ensuremath{\mathrm{X}}}
% Hole Variable Alphabet
\newcommand{\Thvars}{\ensuremath{\Theta}}
% Propositional variables
\newcommand{\Tpvars}{\ensuremath{P}}
% Heap addresses
\newcommand{\Taddr}{\type{Addr}}
% Values
\newcommand{\Tval}{\type{Val}}
% Ranked alphabet
\newcommand{\Tralph}{\ensuremath{\Upsilon}}
% $\Talpha \cup \Tholes$
\newcommand{\Tcalph}{\ensuremath{\Omega}}

% Trees
\newcommand{\Ttree}{\type{Tree}}
% Sequences
\newcommand{\Tseq}{\type{Seq}}
% Heaps
\newcommand{\Theap}{\type{Heap}}
% Terms
\newcommand{\Tterm}{\type{Term}}
% Uniquely-labelled trees
\newcommand{\Tutree}{\type{UTree}}

% Words
\newcommand{\Tword}{\Tcalph^*}
% Forests
\newcommand{\Tfor}{\type{Forest}_{\Talpha,\Tholes}}

% Lists
\newcommand{\Tlist}{\type{Lst}}
% List stores
\newcommand{\Tlsto}{\type{LStore}}

% Abstract states
\newcommand{\Tdata}{\type{D}}

% Contexts
\newcommand{\Tcont}{\type{C}}
% Tree Contexts
\newcommand{\Ttcont}{\sst{\Tcont}{\Ttree}}
% Sequence Contexts
\newcommand{\Tscont}{\sst{\Tcont}{\Tseq}}
% Term Contexts
\newcommand{\Ttercont}{\sst{\Tcont}{\Tterm}}
% Uniquely-labelled tree contexts
\newcommand{\Tutcont}{\sst{\Tcont}{\Tutree}}

% Multi-holed Contexts
\newcommand{\Tmcont}{\ensuremath{\Tcont^\mathit{m}}}
% Multi-holed Tree Contexts
\newcommand{\Tmtcont}{\ensuremath{\Tcont^\mathit{m}_\Ttree}}
% Multi-holed Sequence Contexts
\newcommand{\Tmscont}{\ensuremath{\Tcont^\mathit{m}_\Tseq}}
% Multi-holed Term Contexts
\newcommand{\Tmtercont}{\ensuremath{\Tcont^\mathit{m}_\Tterm}}


% Logical environments
\newcommand{\TLEnv}{\type{LEnv}}
% Propositional interpretations
\newcommand{\TInterp}{\type{Interp}}


% Worlds
\newcommand{\Tworld}{\type{World}}
% Sorts
\newcommand{\Tsort}{\type{Sort}}
% Formulae
\newcommand{\Tformula}{\type{Formula}}

% Ranks
\newcommand{\Trank}{\type{Rank}}

% Duplicator-winning games
\newcommand{\DW}{\type{DW}}
% Spoiler-winning games
\newcommand{\SW}{\type{SW}}


%%%% Mathematics
% Natural numbers (includes 0)
\newcommand{\Tnat}{\mathbb{N}}
% Integers
\newcommand{\Tint}{\mathbb{Z}}
% Booleans
\newcommand{\Tbool}{\type{Bool}}

%%%% Formulae
% (Propositional) formulae
\newcommand{\TF}{\type{F}}
% Data formulae
\newcommand{\TFd}{\type{P}}
% Context Formulae
\newcommand{\TFc}{\type{K}}
% Tree
\newcommand{\TFdt}{\sst{\TFd}{\Ttree}}
% Tree context
\newcommand{\TFct}{\sst{\TFc}{\Ttree}}
% Specifically single holed
\newcommand{\TFsd}{\ensuremath{\TFd^\mathit{s}}}
\newcommand{\TFsc}{\ensuremath{\TFc^\mathit{s}}}
\newcommand{\TFsdt}{\ensuremath{\TFd_\Ttree^\mathit{s}}}
\newcommand{\TFsct}{\ensuremath{\TFc_\Ttree^\mathit{s}}}
% With composition
\newcommand{\TFcd}{\ensuremath{\TFd^\mathit{c}}}
\newcommand{\TFcc}{\ensuremath{\TFc^\mathit{c}}}
\newcommand{\TFcdt}{\ensuremath{\TFd_\Ttree^\mathit{c}}}
\newcommand{\TFcct}{\ensuremath{\TFc_\Ttree^\mathit{c}}}
% Multi-holed
\newcommand{\TFm}{{\ensuremath{\TFc^\mathit{m}}}}
\newcommand{\TFmt}{{\ensuremath{\TFc_\Ttree^\mathit{m}}}}
\newcommand{\TFms}{{\ensuremath{\TFc_\Tseq^\mathit{m}}}}
\newcommand{\TFmter}{{\ensuremath{\TFc_\Tterm^\mathit{m}}}}

%%%% Programs

% Programs
\newcommand{\Tprog}{\type{Prog}}
% Basic command
\newcommand{\Tcommand}{\type{Cmd}}

% Program variables
\newcommand{\Tpgvar}{\type{Var}}
% Variable scope
\newcommand{\Tscope}{\type{Scope}}
% Variable stack
\newcommand{\Tstack}{\type{Stack}}
% Data store
\newcommand{\Tstore}{\type{Store}}
% Program state
\newcommand{\Tstate}{\type{State}}
% Outcomes
\newcommand{\Toutcome}{\type{Outcome}}

% Procedure names
\newcommand{\Tpname}{\type{PName}}
% Procedure definition environment
\newcommand{\Tpdef}{\type{PDef}}

% Expressions
\newcommand{\Texpr}{\type{Expr}}
% Boolean expressions
\newcommand{\Tbexp}{\type{BExp}}

%%%% CAP
% Logical variables
\newcommand{\TLVar}{\type{LVar}}

% Logical state
\newcommand{\TLState}{\type{LState}}
% Shared state
\newcommand{\TSState}{\type{SState}}
% Permission
\newcommand{\TPerm}{\type{Perm}}
% Token
\newcommand{\TTok}{\type{Token}}
% Region identifier
\newcommand{\TRid}{\type{RID}}
% Action name
\newcommand{\TAName}{\type{AName}}
% Action
\newcommand{\TAction}{\type{Action}}
% Action model
\newcommand{\TAMod}{\type{AMod}}

% Abstract predicate name
\newcommand{\TAPName}{\type{APName}}

% Assertion
\newcommand{\Tassn}{\type{Assn}}
\newcommand{\Tbassn}{\type{BAssn}}
\newcommand{\Tiassn}{\type{IAssn}}
\newcommand{\Tfassn}{\type{FAssn}}

\newcommand{\Tinterp}{\type{Interp}}
\newcommand{\Tpenv}{\type{PEnv}}
\newcommand{\Tfenv}{\type{FEnv}}

%% Variables
% =========

% Letters of $\Talpha$
\newcommand{\la}{\mathbf{a}}
\newcommand{\lb}{\mathbf{b}}
\newcommand{\lc}{\mathbf{c}}

% Holes of $\Tholes$
\newcommand{\hx}{x}
\newcommand{\hy}{y}
\newcommand{\hz}{z}
\newcommand{\hxP}{\acute{\hx}}
\newcommand{\hyP}{\acute{\hy}}
\newcommand{\hzP}{\acute{\hz}}


% Variables of $\Thvars$
\newcommand{\hva}{\alpha}
\newcommand{\hvb}{\beta}
\newcommand{\hvc}{\gamma}

% Heap addresses
\newcommand{\hpa}{a}

% Values
\newcommand{\va}{v}
\newcommand{\vb}{w}
\newcommand{\vc}{u}

% Propositional variables
\newcommand{\pvp}{p}
\newcommand{\pvq}{q}

% Trees
\newcommand{\tr}{t}

% Sequences
\newcommand{\sq}{s}

% Heaps
\newcommand{\hp}{h}

% Terms
\newcommand{\tm}{r}

% Lists
\newcommand{\lst}{\mathit{l}}
\newcommand{\lstc}{\mathit{lc}}
\newcommand{\ls}{\mathit{ls}}
\newcommand{\lsc}{\mathit{lsc}}

% Words (confusing with worlds?)
\newcommand{\word}{w}

% Abstract state
\newcommand{\state}{s}
% Contexts
\newcommand{\cont}{c}
\newcommand{\contd}{d}
% Element of $\cid$
\newcommand{\cidc}{i}

% Logical Environments of $\TLEnv$
\newcommand{\lenv}{\sigma}

% Propositional Interpretation of $\TInterp$
\newcommand{\interp}{\iota}

% Context Algebra
\newcommand{\ca}{\mathcal{A}}

% Sort
\newcommand{\sor}{\varsigma}

% Rank
\newcommand{\rank}{r}
% Non-adjunct component
\newcommand{\rn}{m}
% Adjunct component
\newcommand{\rs}{s}
% Label component
\newcommand{\rL}{L}
% Variable component
\newcommand{\rV}{V}

% World
\newcommand{\world}{w}
% Spoiler's world
\newcommand{\swor}{\world^{\mathsf{S}}}
% Duplicator's world
\newcommand{\dwor}{\world^{\mathsf{D}}}

% Operator
\newcommand{\opvar}{\circledast}

% Truth value
\newcommand{\tv}{b}

%%%% Programs

% Program
\newcommand{\cmd}{\mathbb{C}}
% Basic command
\newcommand{\bcmd}{\varphi}
% Procedure definition
\newcommand{\pdef}{\mu}
% Procedure definition environment
\newcommand{\penv}{\gamma}
% Stack
\newcommand{\stack}{s}
% Scope
\newcommand{\scope}{\rho}
% Store
\newcommand{\stor}{\chi}
% Expression
\newcommand{\expr}{\textsl{E}}
% Boolean expression
\newcommand{\bexp}{\textsl{B}}
% Outcome
\newcommand{\oc}{o}

% Procedure specification environment
\newcommand{\pse}{\Gamma}

%%%% Logical formulae
% Propositional formula
\newcommand{\Fp}{F}


% Data formula
\newcommand{\Fdv}{P}
\newcommand{\Fdva}{Q}
\newcommand{\Fdvb}{R}
% Context formula
\newcommand{\Fcv}{K}

% Arbitrary formula
\newcommand{\Fo}{\Fdv}


% Specific data formula
\newcommand{\FSd}{S_\TFd}
\newcommand{\FSc}{S_\TFc}
\newcommand{\FS}{S}

%%%% Abstract modules
% Abstract module
\newcommand{\ama}{\mathbb{A}}
\newcommand{\amb}{\mathbb{B}}

%%%% CAP
% Logical state
\newcommand{\lstate}{l}
% Shared state
\newcommand{\sstate}{s}
% Region identifier
\newcommand{\rid}{r}
% Action name
\newcommand{\aname}{\gamma}
% Permission
\newcommand{\perm}{\phi}
% Action model
\newcommand{\amod}{\zeta}
% Token
\newcommand{\tok}{t}
% Action
\newcommand{\action}{a}
% Abstract predicate name
\newcommand{\apname}{\alpha}


% Interference assertion
\newcommand{\ias}{I}

% Region expression
\newcommand{\ridexp}{R}
% Permission expression
\newcommand{\perexp}{\pi}

% Abstract Predicate environment
\newcommand{\apenv}{\delta}

% Constants
% =========
\RequirePackage{amssymb}

% typeface for constants -- JF
\newcommand{\constant}[1]{\mathbf{#1}}

% null value
\newcommand{\nullref}{\textbf{\textit{nil}}}

% Empty tree
\newcommand{\empt}{\varnothing}
% Context hole
  \newcommand{\holeHELPER}[2]{\raisebox{-.3ex}{$#1#2$}}
\newcommand{\hole}{\protect{\mathpalette{\protect\holeHELPER}{-}}}


% Truth
\newcommand{\truth}{\mathbf{T}}
\newcommand{\falsity}{\mathbf{F}}

\input{notation/operators}
%% Relations
% =========

\RequirePackage{amssymb}
\RequirePackage{eufrak}

% Models
\newcommand{\mods}{\mathrel{\models}}
% Doesn't model
\newcommand{\nmods}{\mathrel{\rlap{$\mkern1.5mu/$}\hbox{$\models$}}}

% Entailment?
\newcommand{\entails}{\subseteq}
% Equivalence
\newcommand{\lequiv}{\equiv}
\RequirePackage{centernot}
\newcommand{\nlequiv}{\centernot\lequiv}

% Rank well-founded relation
\newcommand{\rankwfr}{\mathrel{\lhd}}
% Rank constructor relation
\newcommand{\rankd}[1]{\mathrel{\mathfrak{R}_{#1}}}
% Game move relation
\newcommand{\move}[1]{\mathrel{\mathfrak{M}_{#1}}}
% Simulation preorder
\newcommand{\spo}{\leq}

\RequirePackage{amsmath}

% BNF notation
\newcommand{\Gdef}{\mathrel{\mathop{::}}=}
\newcommand{\Gbar}{\mathbin{\ \big|\ }}

% `is defined to be'
\newcommand{\eqdef}{\stackrel{\mathrm{def}}{=}}

\newcommand{\vect}[1]{\overrightarrow{#1}}

% Semantics
\newcommand{\sem}[1]{\left\llbracket #1 \right\rrbracket}

%%%% Meta syntax
% If and only if
\newcommand{\IFF}{\iff}
% Implies
\newcommand{\IMPLIES}{\implies}
% Or
\newcommand{\OR}{\text{ or }}
% And
\newcommand{\AND}{\text{ and }}
% Not
\newcommand{\NOT}{\text{not }}
% There exists
\newcommand{\EXISTS}[1]{\text{there exists } #1 \text{ s.t.~}}
% There exist
\newcommand{\EXIST}[1]{\text{there exist } #1 \text{ s.t.~}}
% For all
\newcommand{\FORALL}[1]{\text{for all } #1 \text{, }}

% Undefined
% \undef ALREADY EXISTS
\newcommand{\newundef}{\mathit{undefined}}

% Blank
\newcommand{\blank}{{(\cdot)}}

%%%% Logic notations
\newcommand{\CL}{\ensuremath{\mathit{CL}}}
\newcommand{\CLs}{\ensuremath{\CL^\mathit{s}}}
\newcommand{\CLc}{\ensuremath{\CL^\mathit{c}}}
\newcommand{\CLm}{\ensuremath{\CL^\mathit{m}}}

%%%% Games stuff
\newcommand{\Spoiler}{\textsf{Spoiler}}
\newcommand{\Duplicator}{\textsf{Duplicator}}

%%%% Sorts
\newcommand{\mksort}[1]{\mathtt{#1}}
\newcommand{\Sdata}{\mksort{d}}
\newcommand{\Scont}{\mksort{c}}




%% Automata stuff
% --------------


% Language
\newcommand{\Lang}{\mathcal{L}}
\newcommand{\Langi}[1]{\Lang_{#1}}



% Epsilon
\newcommand{\nil}{\varepsilon}
% Label
\newcommand{\lbl}{a}
\newcommand{\lbm}{b}


% Automaton
\newcommand{\aut}{\mathcal{A}}
% State set
\newcommand{\States}{Q}
% Accepting states
\newcommand{\Accs}{A}
% State
\newcommand{\st}{q}
% Initial state
\newcommand{\inist}{e}
% Transition relation
\newcommand{\tra}[1]{f^{#1}}
% $\varepsilon$-closure
\newcommand{\nilclose}{\nil\hbox{-}\mathsf{closure}}

%%% Parameterised
% Automaton
\newcommand{\auti}[1]{\aut_{#1}}
% State set
\newcommand{\Statesi}[1]{\States_{#1}}
% Accepting states
\newcommand{\Accsi}[1]{\Accs_{#1}}
% State
\newcommand{\sti}[1]{\st_{#1}}
% Initial state
\newcommand{\inisti}[1]{\inist_{#1}}
% Transition relation
\newcommand{\trai}[2]{\tra{#1}_{#2}}
% $\varepsilon$-closure
\newcommand{\nilclosei}[1]{\nilclose_{#1}}

% Partial automata
\newcommand{\hataut}{\hat{\aut}}
\newcommand{\hatauti}[1]{\hataut_{#1}}

% Induced mappings
\newcommand{\aint}[1]{\llbracket #1 \rrbracket_{\aut}}
\newcommand{\aintw}[2]{\llbracket #1 \rrbracket_{#2}}
\newcommand{\afint}[1]{\llparenthesis #1 \rrparenthesis_{\aut}}
\newcommand{\afintw}[2]{\llparenthesis #1 \rrparenthesis_{#2}}

% Complementation
\newcommand{\acompl}[1]{\overline{#1}}
% Non-deterministic linear substitution
\newcommand{\nlsub}{\varobslash}
\newcommand{\nllef}{\mathbin{\hbox{$\nlsub$}\mkern-3.1mu \hbox{$-$}\mkern0.3mu}^\exists}
\newcommand{\nlrig}{\mathbin{\hbox{$-$}\mkern-3.1mu \hbox{$\nlsub$}\mkern0.3mu}^\exists}
% Uniform substitution
\newcommand{\usub}{\odot}
\newcommand{\ulef}{\mathbin{\hbox{$\odot$}\mkern-3.1mu \hbox{$-$}\mkern0.3mu}^\exists}
\newcommand{\urig}{\mathbin{\hbox{$-$}\mkern-3.1mu \hbox{$\odot$}\mkern0.3mu}^\exists}

\newcommand{\nusub}{\circledcirc}
\newcommand{\nurig}{\mathbin{\hbox{$-$}\mkern-3.1mu \hbox{$\circledcirc$}\mkern0.3mu}^\exists}

% Reachability
\newcommand{\reach}{\mathsf{reachable}}


% Programs
% ========


\newsavebox{\myarraytmpstrutbox}
\newenvironment{myarray}{\sbox{\myarraytmpstrutbox}{\strutbox}\sbox{\strutbox}{\rule{0pt}{0pt}}\begin{array}}{\end{array}\sbox{\strutbox}{\myarraytmpstrutbox}}


% Program syntax
\newcommand{\psyntax}[1]{\textup{\texttt{#1}}}
% Program variable
\newcommand{\pvs}[1]{\texttt{\textit{#1}}\mspace{-1mu}}
% Procedure name
\newcommand{\pname}[1]{\textup{\texttt{\textbf{#1}}}}

%%%% Expressions

\newcommand{\eP}{\mathbin{\psyntax{+}}}
\newcommand{\eM}{\mathbin{\psyntax{-}}}
\newcommand{\eT}{\mathbin{\psyntax{*}}}
\newcommand{\eLT}{\mathrel{\psyntax{<}}}
\newcommand{\eEQ}{\mathrel{\psyntax{=}}}
\newcommand{\eIMP}{\mathbin{\psyntax{=>}}}
\newcommand{\eFALSE}{\psyntax{false}}


%%%% Programs

% Skip
\newcommand{\pskip}{\psyntax{skip}}
% Assign
\newcommand{\pas}[2]{#1 := #2}
% Sequential composition
\newcommand{\seq}{;}
% If-then-else
\newcommand{\pifelse}[3]{\psyntax{if } #1 \psyntax{ then } #2 \psyntax{ else } #3}
\newcommand{\pifelsev}[3]{\begin{array}{@{\hspace{2ex}}l}\hspace{-2ex}\psyntax{if } #1 \psyntax{ then} \\ #2 \\ \hspace{-2ex}\psyntax{else} \\ #3 \end{array}}
\newcommand{\pifv}[2]{\begin{array}{@{\hspace{2ex}}l}\hspace{-2ex}\psyntax{if } #1 \psyntax{ then} \\ #2 \end{array}}
% While-do
\newcommand{\pwhile}[2]{\psyntax{while } #1 \psyntax{ do } #2}
\newcommand{\pwhilev}[2]{\begin{array}{@{\hspace{2ex}}l}\hspace{-2ex}\psyntax{while } #1 \psyntax{ do } \\ #2 \end{array}}
% Procdef
\newcommand{\procdef}[4]{#1 := #2 ( #3 ) \ \{ #4 \}}
\newcommand{\procdefv}[4]{\begin{array}{@{\hspace{2ex}}l}\hspace{-2ex} #1 := #2 ( #3 ) \ \{ \\ #4 \\ \hspace{-2ex} \} \end{array}}
\newcommand{\procdefvb}[4]{\begin{array}{@{\hspace{2ex}}l}\hspace{-2ex} \pvs{#1} := \pname{#2} ( #3 ) \ \{ \\ #4 \\ \hspace{-2ex} \} \end{array}}
\newcommand{\procdefvc}[3]{\begin{array}{@{\hspace{2ex}}l}\hspace{-2ex} \pname{#1} ( #2 ) \ \{ \\ #3 \\ \hspace{-2ex} \} \end{array}}
% Procedures
\newcommand{\procs}[2]{\psyntax{procs } #1 \psyntax{ in } #2}
\newcommand{\procsv}[2]{\begin{array}{@{\hspace{2ex}}l}\hspace{-2ex}\psyntax{procs} \\ #1 \\ \hspace{-2ex}\psyntax{in} \\ #2 \end{array}}
% Procedure call
\newcommand{\pcall}[3]{\psyntax{call } #1 := #2(#3)}
\newcommand{\pcallo}[2]{\psyntax{call } #1(#2)}
% Local variable declaration
\newcommand{\plocal}[2]{\psyntax{local } #1 \psyntax{ in } #2}
\newcommand{\plocalv}[2]{\begin{array}{@{\hspace{2ex}}l}\hspace{-2ex} \psyntax{local } #1 \psyntax{ in} \\ #2 \end{array}}

% Allocate
\newcommand{\alloc}[2]{\pvs{#1} := \psyntax{alloc}(#2)}
% Dispose
\newcommand{\dispose}[2]{\psyntax{dispose}(#1,#2)}
% Store
\newcommand{\sto}[2]{\left[#1\right] := #2}
% Fetch
\newcommand{\fet}[2]{\pvs{#1} := \left[#2\right]}

%%%% Scope

 \newcommand{\hrpoonyatom}[1]{\vcenter{\offinterlineskip\hbox{$#1\rightharpoonup$}\vskip-.45ex\hbox{$#1\rightharpoondown$}}}

% Scope value
\newcommand{\sceq}{\mathrel{\mathchoice{\hrpoonyatom\displaystyle}{\hrpoonyatom\textstyle}{\hrpoonyatom\scriptstyle}{\hrpoonyatom\scriptscriptstyle}}}
\newcommand{\scval}[2]{ \pvs{#1} \sceq #2 }
% Scope combination
\newcommand{\scc}{\mathrel{\ast}}
% Empty scope
\newcommand{\scemp}{\varnothing}

%%%% Semantics

% $\Tstack \rightarrow \Tscope$
\newcommand{\tosco}[1]{\lfloor #1 \rfloor}

% $\Texpr \rightarrow (\Tscope \rightarrow \Tval)$
\newcommand{\esem}[1]{\mathcal{E}\sem{#1}}
% $\Tbexp \rightarrow \mathcal{P}(\Tscope)$
\newcommand{\psem}[1]{\mathcal{P}\sem{#1}}
% $\Tbexp \rightarrow (\Tscope \rightarrow 2)$
\newcommand{\bsem}[1]{\mathcal{B}\sem{#1}}
% $\Tcommand \rightarrow (\Tscope \rightarrow \powset{\Tscope} \union \Set{\fault})$
\newcommand{\csem}[1]{\mathcal{C}\sem{#1}}
% Big step
\newcommand{\bigto}{\leadsto}
\RequirePackage{centernot}
\newcommand{\nbigto}{\centernot\bigto}
% Fault
\newcommand{\fault}{\lightning}

\DeclareMathOperator{\writeVar}{writeVar}
\DeclareMathOperator{\writeVars}{writeVars}
\DeclareMathOperator{\lookup}{lookup}

\newcommand{\axioms}[1]{\textup{\textsc{Ax}}\sem{#1}}

%%%% Abstract modules
\newcommand{\am}[1]{\mathbb{#1}}
% Heap module
\newcommand{\amH}{\am{H}}
% Tree module
\newcommand{\amT}{\am{T}}
% List module
\newcommand{\amL}{\am{L}}

%%%% Module translations


% Translation
\newcommand{\trl}{\tau}
% Abstraction relation
\newcommand{\ar}{\mathrel{\alpha}}
% Implementation
\newcommand{\imp}[1]{\sem{#1}}
% Predicate translation
\newcommand{\prtr}[1]{\sem{#1}}

% Interface set
\newcommand{\TIF}{\mathcal{I}}
\newcommand{\TIFi}{\TIF_{\mathrm{in}}}
\newcommand{\TIFo}{\TIF_{\mathrm{out}}}
% Interface
\newcommand{\intf}{I}
\newcommand{\intfi}{\mathit{in}}
\newcommand{\intfo}{\mathit{out}}
% Crust parameter set
\newcommand{\TCP}{\mathcal{F}}
% Crust parameter
\newcommand{\crup}{F}
% Data rep
\newcommand{\drep}[2]{\langle\!\langle #1 \rangle\!\rangle^{#2}}
% Context rep
\newcommand{\crep}[3]{\langle\!\langle #1 \rangle\!\rangle^{#2}_{#3}}
% Crust
\newcommand{\crust}[2]{\mathord{\doublecap}_{#1}^{#2}}

\newcommand{\rsem}[1]{\left(\mspace{-4.0mu}\left| #1 \right|\mspace{-3.5mu}\right)}
% Intermediate Data Translation
\newcommand{\idpt}[2]{\rsem{#1}^{#2}}
% Intermediate Context Translation
\newcommand{\icpt}[3]{\rsem{#1}^{#2}_{#3}}


%%%% Predicates

\newcommand{\vsafe}{\mathit{vsafe}}
\newcommand{\bsafe}{\mathit{bsafe}}

%%%% Hoare logic
% Hoare triple
\newcommand{\triple}[3]{\left\{#1\middle\} \ #2 \ \middle\{#3 \right\}}
\newcommand{\vtriple}[3]{\begin{myarray}{@{}c@{}} \left\{#1\right\} \\[\smallskipamount] #2 \\[\smallskipamount] \left\{#3 \right\} \end{myarray}}

\newcommand{\pspec}[3]{#1 : #2 \rightarrowtail #3}


%%%% Hoare logic rules

\newcommand{\mkrule}[1]{\textsc{#1}}
% Axiom
\newcommand{\raxiom}{\mkrule{Axiom}}
% Frame
\newcommand{\rframe}{\mkrule{Frame}}
% Consequence
\newcommand{\rcons}{\mkrule{Cons}}
% Disjunction
\newcommand{\rdisj}{\mkrule{Disj}}
% Skip
\newcommand{\rskip}{\mkrule{Skip}}
% Sequencing
\newcommand{\rseq}{\mkrule{Seq}}
% If
\newcommand{\rif}{\mkrule{If}}
% While
\newcommand{\rwhile}{\mkrule{While}}
% Assignment
\newcommand{\rassgn}{\mkrule{Assgn}}
% Local variable
\newcommand{\rlocal}{\mkrule{Local}}
% Procedure definition
\newcommand{\rpdef}{\mkrule{PDef}}
% Procedure call
\newcommand{\rpcall}{\mkrule{PCall}}
% Procedure weakening
\newcommand{\rpwk}{\mkrule{PWk}}

% Conjunction
\newcommand{\rconj}{\mkrule{Conj}}

%% Concurrent Abstract Predicates
% ==============================

% Well-formedness
\newcommand{\wf}{\mathit{wf}}

% Stability
\newcommand{\stable}[1]{\mathit{stable}\left({#1}\right)}

% Local portion
\newcommand{\locp}[1]{{#1}_\mathrm{L}}
% Shared portion
\newcommand{\shap}[1]{{#1}_\mathrm{S}}
% Heap portion
\newcommand{\hpp}[1]{{#1}_\mathrm{H}}
% Permission portion
\newcommand{\pmp}[1]{{#1}_\mathrm{P}}
% State portion
\newcommand{\stap}[1]{{#1}_\mathrm{R}}
% Action portion
\newcommand{\actp}[1]{{#1}_\mathrm{A}}

% Action model combination
\newcommand{\amplus}{\sqcup}

% Permission combination
\newcommand{\psep}{\mathbin{\oplus}}
% Zero permission
\newcommand{\zperm}{\mathbf{0}_\TPerm}
% Logical state combination
\newcommand{\lssep}{\mathbin{\odot}}
% World combination
\newcommand{\wsep}{\mathbin{\bullet}}
% Zero world
\newcommand{\zworld}{\mathbf{0}_\Tworld}

% Action name
\newcommand{\act}[1]{\textsc{#1}}
% Abstract predicate
\newcommand{\abp}[1]{\mathsf{#1}}

% LState collapse
\newcommand{\lcol}[1]{\lfloor #1 \rfloor}
% Heap collapse
\newcommand{\hcol}[1]{\hpp{\lcol{#1}}}

% Guarantee relation
\newcommand{\Guar}{\mathrel{\mathrm{G}}}
% Construction operation
\newcommand{\Guarc}{\Guar_c}
% Repartitioning guarantee
\newcommand{\Guarr}{\mathrel{\overline{\Guar}}}
% Step-close guarantee
\newcommand{\Guars}{\mathrel{\widehat{\Guar}}}

% Rely relation
\newcommand{\Rely}{\mathrel{\mathrm{R}}}
% Construction
\newcommand{\Relyc}{\Rely_c}

% Predicates
\newcommand{\PRED}{P}
\newcommand{\PREDq}{Q}
\newcommand{\PREDr}{R}
\newcommand{\pred}{p}
\newcommand{\predq}{q}

\newcommand{\spredq}{\mathcal{Q}}

% Abstract predicate axioims
\newcommand{\pde}{\Delta}

% Vector length
\DeclareMathOperator{\len}{len}

% Fixed-point Existence
\newcommand{\fpex}[4]{{#1}\Uparrow{#2},{#3},{#4}}

%%%% Programming language
\newcommand{\atomic}[1]{\left< #1 \right>}
\newcommand{\ppar}{\mathbin{\|}}
\newcommand{\letin}[2]{\psyntax{let } #1 \psyntax{ in } #2}
\newcommand{\pstmts}{c}

\newcommand{\optrans}[1][\eta]{\stackrel{#1}{\rightarrow}}
\newcommand{\noptrans}[1][\eta]{\stackrel{#1}{\centernot{\rightarrow}}}
\newcommand{\safe}[2]{\mathit{safe}_{#1}\left(#2\right)}

%%%% Proof rules
% Atomic
\newcommand{\ratomic}{\mkrule{Atomic}}
% Primitive
\newcommand{\rprim}{\mkrule{Prim}}
% Parallel
\newcommand{\rpar}{\mkrule{Par}}
% Guarantee left
\newcommand{\rguarl}{\mkrule{Guar-L}}
% Guarantee right
\newcommand{\rguarr}{\mkrule{Guar-R}}
% Predicate introduction
\newcommand{\rpredi}{\mkrule{Pred-I}}
% Predicate elimination
\newcommand{\rprede}{\mkrule{Pred-E}}
% Let
\newcommand{\rlet}{\mkrule{Let}}
% Call
\newcommand{\rcall}{\mkrule{Call}}
% Loop
\newcommand{\rloop}{\mkrule{Loop}}
% Choice
\newcommand{\rchoice}{\mkrule{Choice}}
% Existential
\newcommand{\rexist}{\mkrule{Exist}}



%%%%%%%%%%%%%
%%% Fonts %%%

\usepackage{unicode-math}
\defaultfontfeatures{Ligatures=TeX}
\fontspec[SmallCapsFeatures={Letters=SmallCaps}]{Minion Pro}

\setmainfont{Minion Pro}
\setsansfont[Scale=MatchLowercase]{Arial}
\setmathfont[Scale=MatchLowercase]{Asana-Math.otf}
\setmonofont[Scale=MatchLowercase]{Inconsolata}

\linespread{1.1}


%%%%%%%%%%%%%%%%%%%%%
%%% Code listings %%%

\usepackage{minted}
\usemintedstyle{tango}

%Block indent. 1st arg: indent spaces. 2nd arg: thing to indent.
\newcommand{\BIndent}[2]{\begin{tabular}{ll}\IC{#1} \\ #2 \end{tabular}}


\usepackage{sectsty}
