\subsection{Index and set algorithms}

In the previous section,
  the function calls, and their specifications,
  did not reveal anything about
  what \IC{i} looks like \enquote{inside}:
    we don't know how memory is laid out,
    or even what \emph{parts} of memory are allocated for this.
Nor did we specify what a call to \IC{search} (for example) actually \emph{does},
and nor did we specify how long it might take
  (nor even if it halts at all!).

This was necessary and desirable.
It was desirable because it enables the programmer to
  reason about the index as a mathematical construct.\footnotemark
It was necessary in order to separate specification from implementation.
This is particularly important in the case of indexes and sets,
  where a large number of data structures exist that satisfy our \API.
The following is a very partial list:

\footnotetext{
In various terminologies,
  our \IC{Index} API uses \enquote{information hiding},
  also called \enquote{encapsulation}.
}

\VEm
\noindent
\begin{tabular*}{\textwidth}{@{\extracolsep{\fill}}cccc}
Array           & Association list & Radix tree     & van Emde Boas tree \\
Skip list       & Hash table       & Binary tree    & Splay tree         \\
AVL tree        & Bloomier filter  & Red-Black Tree & AA-tree            \\
LLRB-tree       & B-tree           & \BPlusT        & \BStarT            \\
%\ldots             \\
\end{tabular*}
\VEm

After examining of a few of these,
  it will become quickly evident that
  most \emph{index} data structures are simply \emph{set} structures
  with values \enquote{plugged in} next to the key.
An index $\fpfun{I}{K}{V}$ can be represented by
  a set of tuples $(k, v)$ (where $k \in K$ and $v \in V$),
  with the restriction that no two elements have the same $k$ value.
Where the algorithms take this approach,
  for simplicity I instead describe them as implementing a set.

Each approach has its own performance and memory characteristics.
Some have certain restrictions and changes in semantics:
  that the universe of keys must be finite, for instance;
  or that multiple values of the same key can exist.
A good implementation of a \emph{generic} index---one
  that implements the API we described,
    for any universe of keys with an ordering---requires $O(n)$ memory
  and guarantees $O(log(n))$ time, or better, for each operation
  (where $n$ is the number of keys in the index).

Let's now run through the features of a few index data structures.
We will then look at specifying a few more formally.


\subsubsection{Array}

The array is undoubtedly the simplest index algorithm:
  we allocate memory for all members of the universe of keys,
  and initialize all associated values to \IC{null}.
It has $O(1)$ complexity for all operations
  as the location of the key in memory is a simple offset operation.
This is (heavily) paid for in terms of memory,
  the use of which is $O(u)$, where $u$ is the size of the universe of keys.
The array is most commonly used where the key type is \IC{int},
  but can be used with other types if each key maps to a unique integer.
Additionally, the array is only possible where the universe of keys is finite--strings,
  for example, are off-limits.
An array is sensible where
  there is a small allowed key range
  and most or all keys are expected to have a value,
  \eg a map \IC{char}$\pfun$\IC{char} for use in a function \IC{string toUpper(string)}.
Otherwise, an algorithm with a memory complexity of $O(n)$ is better.


\subsubsection{Hash table}

The hash table attempts to solve the array's $O(u)$ memory use
  by choosing an array size
  and storing more than one key at each location.
The key's location in the array (its \NewTerm{bucket})
  is decided by a hash function $h$.
Keys in the same bucket are typically chained in a linked list.
The function $h$ varies from extremely simple
  (\eg $h(k) = k \mod(n)$)
  to complex (\eg cryptographic functions).
The hash table has $O(n)$ worst-case behavior
  in the case that all keys map to the same bucket.
By design, it has poor locality of reference---a good hash function
  spreads similar keys evenly over buckets.
There is no obvious ordered iteration algorithm.


\subsubsection{Association list}

The association list is a linked list of values of type \IC{(K, V)}.
It removes the two disadvantages of the array:
  its size is $O(n)$,
  and the key universe does not have to be finite.
It pays for this with $O(n)$ time complexity
  (except for \IC{insert}, where the new element may be appended to the head of the list).
The association list might be desirable for its simplicity,
  and where the number of elements is expected to be short;
  it is ofted used, for example, in more complex structures including hash tables.

There are many modifications to the basic linked list,
  and these apply to the association list too.
The \enquote{linking} can be done in both directions
  to allow bidirectional traversal.


\subsubsection{Binary search tree}

The Binary Search Tree (\BST) can be seen as an improvement of the linked list,
  arising from the observation that the nodes of an linked list
  can be generalized to contain any number of links.
Following a link in a linked list
  only reduces the set being searched by one element;
  following one link out of a possible two, however,
    potentially halves the number of elements in the candidate set.

The division made at each node is into
  elements smaller than the element being looked at
  and elements larger than it.
Therefore, the key type of a \BST
  must have an \emph{ordering};
  in practise this does not pose a real problem
    as arbitrary data can always be lexicographically ordered.

The division that takes place at every step
  gives the \BST a best-case complexity
  of $Ω(log(n))$ for all operations.\footnotemark  
However, the worst-case scenario,
  in which one of the two sets at every node is the empty set,
  is equivalent to a linked list,
  and thus the \BST still has $O(n)$ complexity.

\footnotetext{
The skip list is another,
  which does this by \enquote{skipping over} more than one node at a time.
\enquote{Balancing} is done by randomizing the height of nodes.
I ignore it because it uses randomization, which is ugly.
}


\subsubsection{Splay tree}

One way to tackle the worst-case of the \BST
  is simply to reduce its relevance
  by ensuring it is a practical corner-case.
This is by no means the case with the standard \BST:
  data is often inserted in an \enquote{almost-sorted} order;
  In a worst-case \BST with the smallest value at the root,
    accessing the largest value still exhibits $O(n)$ time complexity.

The Splay tree tries to make the worst-case less relevant
  by ensuring it does not happen repeatedly.
It does this by, on every operation,
  restructuring the tree to move the latest-accessed element to the root.
This does not change the complexity of the BST operations,\footnotemark
  and instead makes use of the same principle as caching: locality of reference.
The worst-case is still exhibited in other situations:
  after \IC{search}ing for all elements in order,
  the tree will be perfectly unbalanced.

\footnotetext{
Unlike the \BST, it does perform all operations in \emph{amortized} $O(log(n))$;
but its worst-cases are still $O(log(n))$.
}

The Splay tree has the advantage that it is (relatively) simple:
  the data structure is the same as the \BST,
  and the algorithms are the same except for the addition of the \IC{splay} method.
Its disadvantages are significant though:
  it does not provide any worst-case improvement,
  and, unlike other index data structures here,
    its \IC{search} algorithm mutates the tree.


\subsubsection{AVL tree}

The AVL tree improves on the \BST,
  finally reducing all operations to $Θ(log(n))$.
It does this by ensuring that the height of the two subtrees of any node
  differ by at most one.\footnote{\cite{cormen}, p.\ 296.}
We first define the \emph{height} of a node $n$ as

\begin{equation*}
height(n) =
\begin{cases}
0                                        & \text{if $n$ is null,} \\
max(height(n.left), height(n.right)) + 1 & \text{otherwise.}
\end{cases}
\end{equation*}

We then define \emph{balance factor} of a node $n$ as

\begin{equation*}
bf(n) = \begin{cases}
0                                & \text{if $n$ is null,} \\
height(n.left) - height(n.right) & \text{otherwise.}
\end{cases}
\end{equation*}

We can then define an AVL tree recursively as
  \[AVLTree(n) = -1 ≤ bf(n) ≤ 1 ∧ AVLTree(n.left) ∧ AVLTree(n.right).\]
It is the node's balance factor,
  rather than the height,
  that is stored in the node.\footnotemark
(As, by the definition of the AVL tree,
  the balance factor only ever has one of three possible values,
  this can take up only two bits.)


\footnotetext{\cite{haskell_avl}, the most-documented and -commented AVL tree source I know of.}


\subsubsection{B-tree}

The B-tree is primarily motivated by a different problem with the \BST:
  the number of storage reads is $Ω(log_2(n))$.
For media with slow seek times and low granularity
  (such as your hard disk),
  this is too high, and wasteful.
The B-tree's improvement starts by generalizing the binary tree to a $k$-ary tree,
  reducing the number of reads to $Ω(log_k(n))$ (where $k > 2$).
The full solution, however, naturally leads to
  another method to bound the height of the tree in the order of $log(n)$.

A $k$-ary search tree holds $k-1$ elements at each node.
But it is infeasible to demand that each node holds this maximum amount---for
  one thing, we could only store multiples of $k-1$ elements!
This must be relaxed so a node can be less than \enquote{full}.
However, this must not be relaxed so far that
  a node can hold just one (or zero!) elements,
  because the $Ω(log_2(n))$ storage reads then reappears.
The B-tree approach demands that each node\footnotemark is between half-full\footnotemark and full.
More precisely, we can define a B-tree in terms of an integer $t$,
  such that for each node in that tree,
  its number of elements $e$ is such that
  $t-1 ≤ e ≤ 2t-1$.\cite{cormen}

\footnotetext{The root node is an exception, and can be anywhere between empty and full.}
\footnotetext{The \BStarT demands that nodes are at least two-thirds full.}

To illustrate why balance naturally falls out of this,
  consider inserting a value.
Once we have searched for its position,
  we cannot simply create a new node there,
  as this would not satisfy the restriction on the number of elements.
Instead the value must be inserted into an existing leaf node.
If space does not exist at the leaf,
  we create space by splitting the node into two at the same level,
  and inserting the previous median value in the parent using the same technique.
If the splitting does not reach the root, the depth of the tree is unchanged.
Otherwise, the root splits, and the depth increases by one for all leaf nodes.
We can therefore add the additional constraint to the B-tree definition
  that all leaves have the same depth,
  from which we can show that
  the height of a B-tree is $Θ(log(n))$.

Rather than bubble-up the node-splitting,
  most B-tree algorithms actually do this on the downwards traversal
  to the node at which the element will be inserted.
The procedure is simple: every full node visited is split before moving down one level,
  and the median element inserted into the parent.
(We can guarantee that the parent is non-full
  precisely due to the top-down order of the procedure.)
The effect is not equivalent,
  as the top-down procedure splits all full nodes,
  where the bubble-up procedure only splits as many as necessary.
However, balance is still maintained:
  as before, the only time that new levels are created is when the root is split.


\subsubsection{\RBt}

The \RBt inherits ideas from the AVL tree and the B-tree.
It has the same goals and performance guarantees as the AVL tree,
  and uses the same tree mutation to achieve them: rotations.
However, it can also be seen as 
  a mapping of a B-tree of degree $t = 2$ onto a \BST.\footnotemark
This has the advantage over the B-tree that
  adjusting a node does not require copying,
  and less-than-full nodes do not hold unused allocated memory.

\footnotetext{
  \ie, a B-tree where the number of elements $e$ is $1 ≤ e ≤ 3$
    (and therefore the number of links $l$ is $2 ≤ l ≤ 4$).
}

In such a B-tree,
  a 3-node\footnotemark is structured as three element-sized spaces adjacent in memory.
The \RBt instead uses between one and three 1-nodes.
The first 1-node, which I shall call the \enquote{representative-node},
  is representative of the analogous B-tree 3-node.
It is the target of links from parents.
It is accompanied by zero, one or two more 1-nodes, which I shall call \enquote{side-nodes},
  and which contain the other elements in the 3-node.

\footnotetext{I use the term $n$-node to describe a node that can contain up to $n$ elements.}

The representative-node's \IC{left} and \IC{right} pointers have two roles.
Where side-nodes exist,
  the pointers point to these.
(Note this implies that, if the analogous 3-node is full,
  the representative-node will hold the middle element.)
Where a side-node is absent,
  the corresponding pointer instead points directly to the appropriate child representative-node.\footnotemark
A side-node's pointers must point to a child representative-node.
(All pointers may instead be to \enquote{nil} nodes, at which the tree terminates.)

\footnotetext{
Herein lies the space-saving and copy-reducing aspects of the \RBt,
  analogous to the benefits of the \BST over an array.
}

How are we to know whether a pointer leads us to a side-node or a child representative-node?
The approach of Guibas and Sedgewick was to give the pointer a \NewTerm{color}:
  pointers to representative-nodes are black,
  and pointers to side-nodes are red.
In the literature, this has been transformed into the equivalent rule that
  representative-nodes are colored black,
  and side-nodes are colored red.
This has been shortened to simply \enquote{red nodes} and \enquote{black nodes}.

We can now state the properties of the \RBt,
  and relate them to the B-tree:

\begin{enumerate}

\item
\strong{A node is either red or black.}
(A 1-node is either a representative-node or a side-node.)

\item
\strong{The root is black,}
as it is representative of the root 3-node.

\item
\strong{Leaf nodes are black.}
This includes \enquote{nil} nodes
  (which in practise are represented by a single sentinel nil node).

\item
\strong{Red nodes cannot have red children (the \enquote{red rule}).}
This is because they must point to child representative-nodes
  (pointing to another side-node would
  extend the number of elements in a node to more than three).

\item
\strong{For any node,
  all simple paths to a descendant leaf have the same number of black nodes
  (the \enquote{black rule}).}
This number is the node's \NewTerm{black-height}.
This is equivalent to the B-tree rule that all leaves have the same depth.

\end{enumerate}

From the B-tree analogy,
  it should already be evident that the \RBt enforces balance.
However, we can also see this from the last two \RBt properties.
Precisely, we can show that,
  for any node, the longest path to a descendant leaf
  is no more than double the length of the shortest path to a descendant leaf.
If the node has black-height $N$,
  then both the shortest and longest path consist of $N$ black nodes (by property 5), interspersed with red nodes.
The shortest path will contain no red nodes
  (because excising one maintains the \RBt properties and makes the path shorter),
  and therefore has length $N$.
The longest path will contain one red node between each pair of black nodes
  (because introducing another violates property 4),
  and therefore has length $2N-1$.

\RBt operations feel like B-tree operations.
When inserting a new node, we color it red,
  and insert it as we would in a \BST.
Then, if its parent is black, we're done
  (analogous to there being space in a leaf 3-node).
Otherwise, the parent is red, and we have violated the red rule.
We know the grandparent is black.
We check the color of the uncle.
If the uncle is red, we do a color swap on the parent, grandparent, and uncle,
  and move up the tree to look for a possible red violation.
This is analogous to a 3-node being full, splitting it, and moving the median up.
However, if the uncle is black,
  we do a single rotation on the grandparent and some recoloring, and stop.
This is analogous to there being free space \enquote{on the other side} of the 3-node.



\subsubsection{Simplifications of the \RBt}

The standard \RBt is sometimes seen as overly complex.
More precisely,
  the number of violating cases that the algorithms have to rectify
  is seen as too large.
In response, there are simplifications of the standard \RBt.
One is the AA-tree,
  which has the additional requirement that
  red nodes must be right children.
In the B-tree analogy, they are 2-3 trees.
This reduces seven cases to two.

Another is the Left-Leaning Red-Black tree,
  introduced by Sedgewick.
In a similar spirit,
  it reduces the number of cases,
  enforcing that
  no node may have a black left child and a red right child.
This means that, in the B-tree analogy,
  a 3-node only has a single representation in the RB tree.
Unlike the AA-tree, this is still analogous to a 2-3-4 tree.
The main advantage is code simplicity:
  \IC{insert()} takes 33 lines vs 150 lines.

I find these simplifications unsatisfying.
The gain in code simplicity is results in a loss of conceptual clarity.
More importantly, the \RBt algorithms can be drastically shortened
  by tackling the symmetric cases using the same code,
  while remaining semantically equivalent.
A simple technique,
  that of Julienne Walker \cite{jsw},
  uses a two-element array of child pointers
  instead of a \IC{left} and a \IC{right} pointer, like so:

\begin{minted}[]{c++}
struct node {
  int red, data;
  struct node *link[2];
};
\end{minted}

This allows the direction (left or right) to be parameterized,
  encoded as either a \IC{0} or \IC{1}.
The direction can thus be reversed with with the C \IC{!} operator.
