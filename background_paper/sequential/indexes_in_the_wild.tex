\subsection{Indexes in the wild}

Performance of the many index algorithms is well-covered.
However, not all are commonly in use.
A survey of what's-used-where
  should shed light on
  what characteristics are considered most important.

\begin{description}

\item[C++ Standard Template Library]
The GNU implementation
  uses Red-Black trees for sets and maps.\footnotemark

\footnotetext{\cite{libstdcpp_stl_tree_h}.
\enquote{The insertion and deletion algorithms are based on those in Cormen, Leiserson, and Rivest, \textit{Introduction to Algorithms}}.
Used to implement \IC{std::set}, \IC{std::multiset}, \IC{std::map}, and \IC{std::multimap}.
}

\item[Java Class Library]
\IC{TreeMap} uses a \RBt.\footnotemark

\footnotetext{\cite{TreeMap_java}: \enquote{This class provides a red-black tree implementation of the SortedMap interface.  \lacuna The algorithms are adopted from Corman, Leiserson, and Rivest's \textit{Introduction to Algorithms}.}}

\item[.NET standard library]
This uses a \RBt for its \IC{SortedDictionary} and \IC{SortedSet} classes,
  at least in the Mono Project implementation\cite{RBTree_cs}.

\item[Linux kernel]
This uses \RBts for I/O scheduling and for virtual memory.
It uses a pseudo-templating style in C, so there's only one implementation: \IC{lib/rbtree.c}.\footnotemark

\footnotetext{
\cite{linux_kernel_lib}.
In the same directory one will find
  \texttt{btree.c}, implementing a B+tree,
  and \texttt{radix-tree.c}, implementing a radix tree.
}

\item[ext3 filesystem]
This uses a Red-Black tree for directory entries.

\item[Python]
The `CPython' implementation uses a hash.\footnotemark

\footnotetext{\cite{python_dictobject_c}.  In the same directory one will also find \texttt{dictnotes.txt}, an in-depth discussion on optimization of the Python dictionary structure.}

\item[Ruby]
This uses a hashing algorithm for its index.

\item[mobile telephones ...]

\end{description}

It seems that
  most standard libraries
  and low-level codebases
  prefer the \RBt
  to the many alternatives.
