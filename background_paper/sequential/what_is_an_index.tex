\subsection{What is an index?}

An index\footnotemark stores values,
  where each value has a distinct name.
Many data sets fit this description.
For example,
  your mobile's contact list stores \emph{phone numbers},
  and each number is accessed with a person's \emph{name}.
At a larger scale,
  the Web can be seen as a store of \emph{pages},
  where each page has a \emph{URL}.
Even a language can be seen as an index:
  a store of \emph{concepts},
  where each concept can be accessed by a \emph{word}.
More fundamentally,
  any function stores elements from a \emph{range},
  where each element is accessed by a unique number from a \emph{domain}.

\footnotetext{
The same data type also goes by the names
  \emph{map},
  \emph{dictionary},
  \emph{associative array},
  \emph{table},
  and \emph{hash}.
}

\subsubsection{The index \API}

It isn't surprising, then, that
  almost every programming language has an index data type.\cite{wp_mapping}
(Indeed, in some languages, such as Javascript\footnotemark,
  all objects are indexes.)
With its appealingly intuitive interface,
  the programmer can memoize the results of a function,
  model the houses on her street,
  or describe a directed graph.
She might, for another example, cache the Web,
  simply with:\footnotemark

\footnotetext{\cite{ecmascript}, p.\ 2.}

\footnotetext{I use a syntax similar to C++ in my examples.}

\begin{minted}[]{c++}
typedef Index<Url, Page> Webcache;
Webcache cache;

Page get(Url u) {
  Page p = cache.search(u);

  if (p != null && !p.expired()) return p;

  p = http_get(u);

  if (p == null) cache.remove(u);
  else           cache.insert(u, p);

  return p;
}
\end{minted}

This small example illustrates the entire \API.
Let's look at it a bit more formally.
The index is an \emph{abstract} data type,\footnote{Also called a \emph{container type}.}
  and cannot be used as-is;
  the \emph{key} and the \emph{value} must first be given a type.
The resulting \emph{concrete data type} can then be instantiated.
(In the above,
  \IC{Index} is the abstract data type,
  and \IC{Webcache} is the concrete data type,
  with \IC{Url}s as keys
  and \IC{Page}s as values.)
We say that an index `maps \IC{K}s onto \IC{V}s'
  if its key type is \IC{K}
  and its value type is \IC{V}.
Given an index \IC{i}
  that maps \IC{K}s onto \IC{V}s,
  we can specify the semantics of operations on it.
In the following,
  the variable \IC{key} is of type \IC{K},
  and the variable \IC{value} is of type \IC{V}:

\begin{description}

%%%%% search
\item[\IC{value = i.search(key);}\footnotemark]
If there is a value associated with \IC{key} in \IC{i},
  \IC{value} will be that value.
Otherwise, \IC{value} will be \IC{null}.\footnotemark

\footnotetext{
The \IC{search} function is also called
  \IC{find}, \IC{fetch}, \IC{read}, and \IC{get}.
}

\footnotetext{
Note that with a \IC{null} type,
  an index can be seen as a \emph{total} function.
}

%%%%% insert
\item[\IC{i->insert(key, value);}\footnotemark]
Subsequent calls to \IC{i->search(key)} will return \IC{value}
  (until subsequent calls to \IC{i->insert(k, —)} or \IC{i->remove(k)},
   where \IC{k == key}).

\footnotetext{
The \IC{insert} function is also called
  \IC{store}, \IC{set}, \IC{save}, and \IC{add}.
}

%%%%% remove
\item[\IC{i->remove(key);}\footnotemark]
Subsequent calls to \IC{i.read(key)} will return \IC{null}
  (until subsequent calls to \IC{i->insert(k, —)} or \IC{i->remove(k)},
   where \IC{k == key}).

\footnotetext{
The \IC{remove} function is also called
  \IC{delete}.
It may also be absent,
  in favour of \IC{insert(k, null)}
  (suggesting the `total function' interpretation).
}
\end{description}

\subsubsection{Formalizing behavior: a Set \ADT}

A correct index implementation will satisfy the behavior described above.
To \emph{prove} that one does so, however,
  we shall have to describe the behavior more precisely.

The basic tool we use to do this is the concept of of \emph{pre- and post-conditions}.\footnotemark
In the context of writing a specification,
  these let us specify the possible valid states of the program before executing a command,
  and what the state of the program will be after execution of it.
For example, we might state that if $x=2$ prior to executing \IC{x=square(x)},
  then $x=4$ afterwards.
This would be written as the \enquote{Hoare triple}
  \Triple{x=2}{x=square(x)}{x=4}.\footnotemark
More complex \enquote{commands},
  such as the operations on an index,
  can be specified in the same way:
  \Triple{\ldots}{i->insert(k, v)}{\ldots}.
For an operation like this,
  the precondition must capture
  the valid states of the variables \IC{i}, \IC{k} and \IC{v}.

\footnotetext{
This is a central concept of Hoare logic.\cite{hoare}
There is a lot more to basic Hoare logic,
  but here I am only specifying behaviour,
  not proving that anything satisfies it.
}

\footnotetext{
Actually, the triple makes no assertions as to whether the command terminates,
  so the assertion made is strictly that
  the postcondition will hold \emph{if and when} the command terminates.
A second syntax is used for assertions that the command terminates:
  \TermTriple{x=2}{x=square(x)}{x=4}.
I do not concern myself with this here or in the specification.
}

However, \IC{i}, a pointer to a valid index in memory,
can be a problematic thing to capture.
First, we don't actually \emph{know} what a valid index looks like,
  because, as in the above descriptions,
  we do not want to concern ourselves with these low-level details
  when describing behavior.
Second, even if we did want to describe memory layout,
  the details of this would be too complex to be practicable in a specification.

The solution is simply to skip over this problem at this stage,
  and instead use an \NewTerm{abstract predicate}
  to describe what \IC{i} represents to the programmer using it,
  namely, \enquote{a pointer to something representing an index containing…}.
This predicate over \IC{i} is written as $\Pred{index}{\IC{i}, I}$.

A bit more notation is necessary to describe the mappings in an index.
However, at this point and with familiar set notation,
  we can already fully specify a similar abstract data type: the \emph{set}.
This will be useful, because a data structure representing an index
  can be constructed by minimal modification of a set data structure.

The following functions for a Set \ADT, and their specifications, should be intuitive.
We can \IC{search} for, \IC{insert}, and \IC{remove} elements from the set,
  just as we can do with mappings in an index.
The pre- and postconditions simply describe
  how and whether the functions mutate the set,
  and how information is extracted from the structure.

\VEm
\noindent
\begin{tabular*}{\textwidth}{l l l}

% search where key is in the set
\Cond{\Pred{set}{\IC{s}, S} \Fand \IC{el} \in S}  &
\IC{exists = s->search(el);}  &
\Cond{\Pred{set}{\IC{s}, S} \Fand \IC{el} \in S \Fand \IC{exists} = \truth}
\\[0.5em]

% search where key is not in the set
\Cond{\Pred{set}{\IC{s}, S} \Fand \IC{el} \notin S}  &
\IC{exists = s->search(el);}  &
\Cond{\Pred{set}{\IC{s}, S} \Fand \IC{el} \notin S \Fand \IC{exists} = \falsity}
\\[0.5em]

% insert
\Cond{\Pred{set}{\IC{s}, S}}  &
\IC{s->insert(el);}  &
\Cond{\Pred{set}{\IC{s}, S \union \Set{\IC{el}}}}
\\[0.5em]

% remove
\Cond{\Pred{set}{\IC{s}, S}}  &
\IC{s->remove(el);}  &
\Cond{\Pred{set}{\IC{s}, S \setminus \Set{\IC{el}}}}
\\

\end{tabular*}


\subsubsection{An index \API specification}

Specifying an index \ADT is only marginally more complex.
As with the set$S$, the index $I$ is now purely logical,
  and we can describe it mathematically without concern for memory layout or local variables.
$I$ is a mapping from a set of keys
  (which is a subset of the universe of keys)
  to a universe of values.
In other words it is a \NewTerm{finite partial function},
  written as
  $I : \mathsf{Keys} \fpfun \mathsf{Values}$.

As $\mathsf{Keys}$ is just a set,
  we can express whether or not a key is in the index
  using ordinary set notation:
  $x \in \KeysOf{I}$ states that
  there is a mapping in the index from $x$ to some value.
As $I$ is a function,
  we can express a specific mapping
  using ordinary function notation:
  $I(x) = y$ states that
  there is a mapping in the index from $x$ to $y$.

With just a couple more shorthand notations,
  we can then express the desired behavior of the index.
First, the \NewTerm{function update} notation,
  $\fupd{I}{k}{v}$,
  denotes the finite partial function $I$ with the modification that $I(k) = v$.
Second, the \NewTerm{set difference} notation $I \setminus \Set{k}$
  is a shorthand for $I$ with the removal of the key $k$ from $\KeysOf{I}$.

The specification of the desired index behavior follows.

\VEm
\noindent
\begin{tabular*}{\textwidth}{l l l}

% search where key is in the index
\Cond{\Pred{index}{\IC{i}, I} \Fand I(\IC{k}) = v\prime}  &
\IC{v = i->search(k);}  &
\Cond{\Pred{index}{\IC{i}, I} \Fand I(\IC{k}) = v\prime \Fand \IC{v} = v\prime}
\\[0.5em]

% search where key is not in the index
\Cond{\Pred{index}{\IC{i}, I} \Fand \IC{k} \notin \KeysOf{I}}  &
\IC{v = i->search(k);}  &
\Cond{\Pred{index}{\IC{i}, I} \Fand \IC{k} \notin \KeysOf{I} \Fand \IC{v} = \nullref}
\\[0.5em]

% insert where key is in the index
\Cond{\Pred{index}{\IC{i}, I} \Fand I(\IC{k}) = v\prime}  &
\IC{i->insert(k, v);}  &
\Cond{\Pred{index}{\IC{i}, I} \Fand I(\IC{k}) = v\prime}
\\[0.5em]

% insert where key is not in the index
\Cond{\Pred{index}{\IC{i}, I} \Fand \IC{k} \notin \KeysOf{I}}  &
\IC{i->insert(k, v);}  &
\Cond{\Pred{index}{\IC{i}, J) \Fand \fupd{I}{\IC{k}}{\IC{v}}}}
\\[0.5em]

% remove
\Cond{\Pred{index}{\IC{i}, I} }  &
\IC{i->remove(k);}  &
\Cond{\Pred{index}{\IC{i}, I \setminus \Set{\IC{k}} }}
\\

\end{tabular*}
\VEm



