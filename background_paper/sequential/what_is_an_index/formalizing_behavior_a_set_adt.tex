\subsubsection{Formalizing behavior: a Set \ADT}

A correct index implementation will satisfy the behavior described above.
To \emph{prove} that one does so, however,
  we shall have to describe the behavior more precisely.

The basic tool we use to do this is the concept of of \emph{pre- and post-conditions}.\footnotemark
In the context of writing a specification,
  these let us specify the possible valid states of the program before executing a command,
  and what the state of the program will be after execution of it.
For example, we might state that if $x=2$ prior to executing \IC{x=square(x)},
  then $x=4$ afterwards.
This would be written as the \enquote{Hoare triple}
  \Triple{x=2}{x=square(x)}{x=4}.\footnotemark
More complex \enquote{commands},
  such as the operations on an index,
  can be specified in the same way:
  \Triple{\ldots}{i->insert(k, v)}{\ldots}.
For an operation like this,
  the precondition must capture
  the valid states of the variables \IC{i}, \IC{k} and \IC{v}.

\footnotetext{
This is a central concept of Hoare logic.\cite{hoare}
There is a lot more to basic Hoare logic,
  but here I am only specifying behaviour,
  not proving that anything satisfies it.
}

\footnotetext{
Actually, the triple makes no assertions as to whether the command terminates,
  so the assertion made is strictly that
  the postcondition will hold \emph{if and when} the command terminates.
A second syntax is used for assertions that the command terminates:
  \TermTriple{x=2}{x=square(x)}{x=4}.
I do not concern myself with this here or in the specification.
}

However, \IC{i}, a pointer to a valid index in memory,
can be a problematic thing to capture.
First, we don't actually \emph{know} what a valid index looks like,
  because, as in the above descriptions,
  we do not want to concern ourselves with these low-level details
  when describing behavior.
Second, even if we did want to describe memory layout,
  the details of this would be too complex to be practicable in a specification.

The solution is simply to skip over this problem at this stage,
  and instead use an \NewTerm{abstract predicate}
  to describe what \IC{i} represents to the programmer using it,
  namely, \enquote{a pointer to something representing an index containing…}.
This predicate over \IC{i} is written as $\Pred{index}{\IC{i}, I}$.

A bit more notation is necessary to describe the mappings in an index.
However, at this point and with familiar set notation,
  we can already fully specify a similar abstract data type: the \emph{set}.
This will be useful, because a data structure representing an index
  can be constructed by minimal modification of a set data structure.

The following functions for a Set \ADT, and their specifications, should be intuitive.
We can \IC{search} for, \IC{insert}, and \IC{remove} elements from the set,
  just as we can do with mappings in an index.
The pre- and postconditions simply describe
  how and whether the functions mutate the set,
  and how information is extracted from the structure.

\VEm
\noindent
\begin{tabular*}{\textwidth}{l l l}

% search where key is in the set
\Cond{\Pred{set}{\IC{s}, S} \Fand \IC{el} \in S}  &
\IC{exists = s->search(el);}  &
\Cond{\Pred{set}{\IC{s}, S} \Fand \IC{el} \in S \Fand \IC{exists} = \truth}
\\[0.5em]

% search where key is not in the set
\Cond{\Pred{set}{\IC{s}, S} \Fand \IC{el} \notin S}  &
\IC{exists = s->search(el);}  &
\Cond{\Pred{set}{\IC{s}, S} \Fand \IC{el} \notin S \Fand \IC{exists} = \falsity}
\\[0.5em]

% insert
\Cond{\Pred{set}{\IC{s}, S}}  &
\IC{s->insert(el);}  &
\Cond{\Pred{set}{\IC{s}, S \union \Set{\IC{el}}}}
\\[0.5em]

% remove
\Cond{\Pred{set}{\IC{s}, S}}  &
\IC{s->remove(el);}  &
\Cond{\Pred{set}{\IC{s}, S \setminus \Set{\IC{el}}}}
\\

\end{tabular*}

