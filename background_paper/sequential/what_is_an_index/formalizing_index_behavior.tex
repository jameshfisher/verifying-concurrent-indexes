\subsubsection{An index \API specification}

Specifying an index \ADT is only marginally more complex.
As with the set$S$, the index $I$ is now purely logical,
  and we can describe it mathematically without concern for memory layout or local variables.
$I$ is a mapping from a set of keys
  (which is a subset of the universe of keys)
  to a universe of values.
In other words it is a \NewTerm{finite partial function},
  written as
  $I : \mathsf{Keys} \fpfun \mathsf{Values}$.

As $\mathsf{Keys}$ is just a set,
  we can express whether or not a key is in the index
  using ordinary set notation:
  $x \in \KeysOf{I}$ states that
  there is a mapping in the index from $x$ to some value.
As $I$ is a function,
  we can express a specific mapping
  using ordinary function notation:
  $I(x) = y$ states that
  there is a mapping in the index from $x$ to $y$.

With just a couple more shorthand notations,
  we can then express the desired behavior of the index.
First, the \NewTerm{function update} notation,
  $\fupd{I}{k}{v}$,
  denotes the finite partial function $I$ with the modification that $I(k) = v$.
Second, the \NewTerm{set difference} notation $I \setminus \Set{k}$
  is a shorthand for $I$ with the removal of the key $k$ from $\KeysOf{I}$.

The specification of the desired index behavior follows.

\VEm
\noindent
\begin{tabular*}{\textwidth}{l l l}

% search where key is in the index
\Cond{\Pred{index}{\IC{i}, I} \Fand I(\IC{k}) = v\prime}  &
\IC{v = i->search(k);}  &
\Cond{\Pred{index}{\IC{i}, I} \Fand I(\IC{k}) = v\prime \Fand \IC{v} = v\prime}
\\[0.5em]

% search where key is not in the index
\Cond{\Pred{index}{\IC{i}, I} \Fand \IC{k} \notin \KeysOf{I}}  &
\IC{v = i->search(k);}  &
\Cond{\Pred{index}{\IC{i}, I} \Fand \IC{k} \notin \KeysOf{I} \Fand \IC{v} = \nullref}
\\[0.5em]

% insert where key is in the index
\Cond{\Pred{index}{\IC{i}, I} \Fand I(\IC{k}) = v\prime}  &
\IC{i->insert(k, v);}  &
\Cond{\Pred{index}{\IC{i}, I} \Fand I(\IC{k}) = v\prime}
\\[0.5em]

% insert where key is not in the index
\Cond{\Pred{index}{\IC{i}, I} \Fand \IC{k} \notin \KeysOf{I}}  &
\IC{i->insert(k, v);}  &
\Cond{\Pred{index}{\IC{i}, J) \Fand \fupd{I}{\IC{k}}{\IC{v}}}}
\\[0.5em]

% remove
\Cond{\Pred{index}{\IC{i}, I} }  &
\IC{i->remove(k);}  &
\Cond{\Pred{index}{\IC{i}, I \setminus \Set{\IC{k}} }}
\\

\end{tabular*}
\VEm


